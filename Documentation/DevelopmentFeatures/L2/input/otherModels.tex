
%%%%%%%%%%%%%%%%%%%%%%%%%%%%%%%%%%%%%%%%%%%%%%%%%%%%%%%%%%%%%%%%%%%%%%%%
\chapter{Other observation models}
\label{ch:otherModels}

There is a large family of models which cannot be expressed as 
\emph{structured} models. Here examples described in detail in 
this chapter.

\begin{itemize}
\item 
Non-standard combined log-transformed model
\item 
BQL residual model
\item 
Variability in residual error 
\begin{itemize}
\item 
IIV in residual error magnitude 
\item 
IOV in residual error magnitude 
\item 
IRV of the residual error 
\item 
influence of regressors
\end{itemize}
\item 
Joint residual error between multiple observations 
\item 
Two (or more) types of measurements error model 
%\item 
%Prior on residual error magnitude -- covered by PharmML 0.7.3 \marginpar{\textcolor{red}{ToDo}}
%\item 
%Mixture model for residual error \marginpar{\textcolor{red}{ToDo}}
%\item 
%Two (or more) types of observations model 
%\item 
%Serial correlation -- not supported right now
\end{itemize}

% left overs
%1.3 & Combination of L2 data items with likelihood based inclusion of BLQ data & -- & $\star$  \\
%1.6 & Residual error magnitude varying with derivatives of functions w.r.t. to time and parameters & -- & $\star$ \\
%1.9 & Flexible errors-in-variables models & $\surd$ & $\star\star$ \\
%1.10 & Transformation of residual error variables & -- & $\star\star\star$ \\


%---------------------------------------------3.1-----------------------------------------------------------------------
\section{Non-standard combined log-transformed model} 

The following variance model was proposed 
specifically to get the same error structure for log-transformed data as 
in case of the 'combined constant and proportional model 2' for the 
untransformed data (see
\href{http://www.cognigencorp.com/nonmem/nm/99apr232002.html}{[NMusers] forum} discussion).

\bigskip
Model definition:
\begin{eqnarray}
&& \log(y_{ij}) = \log(f_{ij}) + \sqrt{a^2 + b^2/f_{ij}^2}\,\epsilon_{ij}; \quad \epsilon_{ij} \sim N(0,1); \quad \mathit{var}(y_{ij}) = a^2 + b^2/f_{ij}^2. \nonumber
\end{eqnarray}

\bigskip
% NONMEM
\begin{lrbox}{\lstbox}\begin{minipage}{16cm}
NONMEM 
\begin{lstlisting}[frame=single,language=NM]
$ERROR
ADD = THETA(1)
PROP = THETA(2)
Y=LOG(F)+SQRT(PROP**2+ADD**2/F**2)*EPS(1)
$THETA 1; ADD
	   1; PROP
$SIGMA 1 FIX
\end{lstlisting}   
\end{minipage}\end{lrbox}
\usebox\lstbox


% MLXTRAN
\begin{lrbox}{\lstbox}\begin{minipage}{16cm}
MLXTRAN
\begin{lstlisting}[frame=single,language=MLX]
[LONGITUDINAL]
input = {..., a, b}

EQUATION:
F = ...
G = sqrt(a^2 + b^2/F^2)

DEFINITION:
Y = {distribution=logNormal, prediction=F, sd=G}
\end{lstlisting}   
\end{minipage}\end{lrbox}
\usebox\lstbox


%---------------------------------------------3.2-----------------------------------------------------------------------
\section{BQL residual model} 

This model was proposed by \cite{Beal:2001}: \emph{Logarithmically 
transformed observations are used, and are modeled with the 
homogeneous variance model, when the distributions of the untransformed 
observations may be positively skewed, but their cv's seem to be constant. 
However, there may be observations whose pharmacokinetic
predictions become theoretically small, but both their central tendency
and variance seem to remain constant and above certain levels
(assuming that the assay is accurate, this can only happen when the kinetics
are misspecified), in which case another useful model for the logarithmically
transformed observations is given by the formula below where M is an extra 
positively constrained parameter and $\epsilon_1$ and $\epsilon_2$ are random errors}.\\

\smallskip
Model definition:
\bigskip
\begin{eqnarray}
\log(y_{ij}) =  \log(f_{ij}+M) + \frac{f_{ij}}{f_{ij}+M} \epsilon_{1,ij} + \frac{M}{f_{ij}+M} \epsilon_{2,ij}; \quad \epsilon_{1,ij}, \epsilon_{2,ij}\sim N(0,1); \quad \mathit{var}(y_{ij}) = \frac{f_{ij}^2 + M^2}{(f_{ij}+M)^2}\nonumber
\end{eqnarray}
for uncorrelated $\epsilon_{1,ij}$ and $\epsilon_{2,ij}$.
\bigskip

Alternative model definition:
\begin{eqnarray}
y_{ij} \sim \mathcal{N}\Big( \log(f_{ij}+M), \frac{f_{ij}^2 + M^2}{(f_{ij}+M)^2} \Big) \nonumber
\end{eqnarray}

\bigskip
% NONMEM
\begin{lrbox}{\lstbox}\begin{minipage}{16cm}
NMTRAN
(from \url{www.cognigencorp.com/nonmem/nm/99apr242002.html}):
\begin{lstlisting}[frame=single,language=NM]
$ERROR
M=THETA(1)
Y=LOG((F)+M)+(F*EPS(1))/(F+M)+(M*EPS(2))/(F+M)
$THETA (0,0.1); M
$SIGMA 1 FIX
$SIGMA 1 FIX
\end{lstlisting}   
\end{minipage}\end{lrbox}
\usebox\lstbox



% MLXTRAN
\begin{lrbox}{\lstbox}\begin{minipage}{16cm}
MLXTRAN
\begin{lstlisting}[frame=single,language=MLX]
[LONGITUDINAL]
input = {...,M}

EQUATION:
F = ...
FM = F + M
G = sqrt(F^2 + M^2)/FM

DEFINITION:
Y = {distribution=logNormal, prediction = FM, sd=G}
\end{lstlisting}   
\end{minipage}\end{lrbox}
\usebox\lstbox


%%%%%%%%%%%%%%%%%%%%%%%%%%%%%%%%%%%%%%%%%%%%%%%%%%%%%%%%%%%%%%%%%%%%%%%%
%---------------------------------------------------------------------------3.3.1---------------------------------------------------------------------------------------
\section{Variability in residual error magnitude and parameters}
\label{otherModels_model1}

%---------------------------------------------------------------------------3.3.1---------------------------------------------------------------------------------------
\subsection{IIV of the residual error magnitude -- \emph{aka} ETA-on-EPS}

Model definition:\\
Let $g_{ij}$ be the individual log-normally distributed standard deviation of the residual error
\begin{eqnarray}
&& g_{ij} = g_{typ,ij} \exp(\eta_i); \quad \eta_{i} \sim \mathcal{N}(0,\omega_g) \nonumber
\end{eqnarray}
with $g_{typ,ij}$ as the typical value, see section \ref{subsec:IIVonResidualError}, e.g.
\begin{eqnarray}
&& g_{typ,ij} = \sqrt{a^2 + b^2 f_{ij}^2} \nonumber
\end{eqnarray}
The Gaussian observation model for untransformed data reads then
\begin{eqnarray}
&& y_{ij} = f_{ij} + g_{ij} \epsilon_{ij}, \quad \epsilon_{ij} \sim \mathcal{N}(0,1); \quad var(y_{ij}) = g_{ij}^2	 \nonumber
\end{eqnarray}


\bigskip
% NONMEM
\begin{lrbox}{\lstbox}\begin{minipage}{16cm}
NMTRAN
\begin{lstlisting}[frame=single,language=NM]
$ERROR
ADD=THETA(1)
PROP=THETA(2)
IPRED = F
G = SQRT(ADD**2+PROP**2*IPRED**2) * EXP(ETA(1))
Y = IPRED + G*EPS(1)
$THETA 1; ADD
	 1; PROP
$OMEGA 1
$SIGMA 1 FIX
\end{lstlisting}   
\end{minipage}\end{lrbox}
\usebox\lstbox


\begin{lrbox}{\lstbox}\begin{minipage}{16cm}
MLXTRAN
\begin{lstlisting}[frame=single,language=MLX]
[INDIVIDUAL]
input={..., omega_G}

DEFINITION:
eta_G = {distribution=normal, mean=0, sd=omega_G}

[LONGITUDINAL]
input={..., a, b, eta_G}

EQUATION:
F = ...
G = sqrt(a^2 + b^2*F^2)*eta_G

DEFINITION:
Y = {distribution=normal, prediction=F, sd=G}
\end{lstlisting}   
\end{minipage}\end{lrbox}
\usebox\lstbox


%---------------------------------------------------------------------------3.3.2---------------------------------------------------------------------------------------
\subsection{IIV of the residual error parameters}
Alternative IIV ....
\label{subsec:altIIV}
The models above illustrates modelling IIV of the residual error magnitude via the 
standard deviation function, $G$. An alternative is to assign IIV to the parameters 
of the residual error as shown in this example.

\bigskip
Model definition:
\begin{eqnarray}
&& a_i = a_{pop}\, \exp(\eta_{a,i}); \quad \eta_{a,i} \sim \mathcal{N}(0,\omega_a^2) \nonumber \\
&& b_i = b_{pop}\, \exp(\eta_{b,i}); \quad \eta_{b,i} \sim \mathcal{N}(0,\omega_b^2) \nonumber \\
&& g_{ij} = \sqrt{a_i^2 + b_i^2 f_{ij}^2} \nonumber \\
&& y_{ij} = f_{ij} + g_{ij} \epsilon_{ij}; \quad \epsilon_{ij} \sim \mathcal{N}(0,1); \quad var(y_{ij}) = g^2_{ij}	\nonumber
\end{eqnarray}

\bigskip
% NONMEM
\begin{lrbox}{\lstbox}\begin{minipage}{16cm}
NMTRAN
\begin{lstlisting}[frame=single,language=NM]
$ERROR
ADD_pop = THETA(1)
PROP_pop = THETA(2)
ADD = ADD_pop*ETA(1)
PROP = PROP_pop*ETA(2)
G = SQRT(ADD**2 + PROP**2*F**2)
Y = F + G*EPS(1)
$THETA 1; ADD_pop
	 1; PROP_pop
$OMEGA 1
$OMEGA 1
$SIGMA 1 FIXED
\end{lstlisting}   
\end{minipage}\end{lrbox}
\usebox\lstbox


% MLXTRAN
\begin{lrbox}{\lstbox}\begin{minipage}{16cm}
MLXTRAN
\begin{lstlisting}[frame=single,language=MLX]
[INDIVIDUAL]
input = {..., a_pop, b_pop, omega_a, omega_b}

DEFINITION:
a = {distribution=logNormal, prediction=a_pop,sd=omega_a}
b = {distribution=logNormal, prediction=b_pop,sd=omega_b}

[LONGITUDINAL]
input = {..., a, b}

EQUATION:
F = ...
G = sqrt(a^2+b^2*F^2)

DEFINITION
Y = {distribution=normal, mean=F, sd=G}
\end{lstlisting}   
\end{minipage}\end{lrbox}
\usebox\lstbox


%---------------------------------------------------------------------------3.3.3---------------------------------------------------------------------------------------
\subsection{IOV of the residual error magnitude}

Here the magnitude of the residual error varies with occasions, expressed with subscript $k$.

\bigskip
Model definition:

\begin{eqnarray}
&&g_{typ,ij} = \sqrt{a^2 + b^2 f_{ij}^2} \nonumber \\
&&g_{ijk} = g_{typ,ij} \exp(\eta_{ik}) \nonumber 
\end{eqnarray}
with $\eta_{ik} \sim \mathcal{N}(0,\gamma^2)$. 

\bigskip

The Gaussian observation model for untransformed data reads then
\begin{eqnarray}
&& y_{ijk} \sim \mathcal{N}(f_{ij},g_{ijk}^2) \nonumber \\
or && y_{ijk} = f_{ij} + g_{ijk} \epsilon_{ij}; \quad \epsilon_{ij} \sim \mathcal{N}(0,1); \quad var(y_{ijk}) = g^2_{ijk}	 \nonumber
\end{eqnarray}

\bigskip 
% NONMEM
\begin{lrbox}{\lstbox}\begin{minipage}{16cm}
NMTRAN (with two occasions indicated with values 1 or 2 in the OCC column)
\begin{lstlisting}[frame=single,language=NM]
$ERROR
ADD = THETA(1)
PROP = THETA(2)
IF(OCC.EQ.1) IOV = ETA(1)
IF(OCC.EQ.2) IOV = ETA(2)
G = SQRT(ADD**2+PROP**2*IPRED**2) * EXP(IOV)
Y = IPRED + G*EPS(1)

$THETA 1; ADD
	 1; PROP
$OMEGA BLOCK (1) 0.5
$OMEGA BLOCK (1) SAME
$SIGMA 1 FIX
\end{lstlisting}   
\end{minipage}\end{lrbox}
\usebox\lstbox

% ORIGINAL CODE
%$ERROR
%; Example IOV on EPS
%FLAG1 = 0
%FLAG2 = 0
%IF(FLAG.EQ.1) FLAG1 = 1
%IF(FLAG.EQ.0) FLAG2 = 1
%IOV = FLAG1*ETA(1) + FLAG2*ETA(2)
%W = SQRT(THETA(1)**2+THETA(2)**2*IPRED*IPRED) * EXP(IOV)
%Y = IPRED + W*EPS(1)
%
%$THETA 1
%	   1
%$OMEGA BLOCK (1) 0.5
%$OMEGA BLOCK (1) SAME
%$SIGMA 1 FIX
 
\begin{lrbox}{\lstbox}\begin{minipage}{16cm} 
MLXTRAN \\
%\textcolor{red}{version 1}\\
 (with two occasions indicated with values 1 or 2 in the OCC column of type REG.  
Note that this cannot change the connectivity of records in the dataset in terms of time values.)
 
\begin{lstlisting}[frame=single,language=MLX]
[INDIVIDUAL]
input={..., omega}

DEFINITION:
eta_G1 = {distribution=normal, mean=0,sd=omega}
eta_G2 = {distribution=normal, mean=0,sd=omega}

[LONGITUDINAL]
input={..., eta_G1, eta_G2, a, b}
regressor={OCC}

EQUATION:
F = ...
if OCC==1
   IOV=eta_G1
else
   IOV=eta_G2
end
G = sqrt(a^2+b^2*F^2) * exp(IOV)

DEFINITION:
Y = {distribution=normal, mean=F, sd=G}
\end{lstlisting}   
\end{minipage}\end{lrbox}
\usebox\lstbox
%\marginpar{\textcolor{red}{correct \\ use of \\ regressor? \\}}
%
%\marginpar{\textcolor{red}{is \\ version 1 \\ correct?\\ \\ \\}}


\begin{lrbox}{\lstbox}\begin{minipage}{16cm} 
MLXTRAN \\
%\textcolor{red}{version 2}
\begin{lstlisting}[frame=single,language=MLX]
[INDIVIDUAL]
input={..., omega}

DEFINITION:
eta_G1 = {distribution=normal, mean=0,sd=omega}
eta_G2 = {distribution=normal, mean=0,sd=omega}

[LONGITUDINAL]
input={..., eta_G1, eta_G2, a, b}
regressor={OCC}

EQUATION:
F = ...
G_typ = sqrt(a^2 + b^2*F^2)
if OCC==1
   G=G_typ*exp(eta_G1)
else
   G=G_typ*exp(eta_G2)
end

DEFINITION:
Y = {distribution=normal, mean=F, sd=G}
\end{lstlisting}   
\end{minipage}\end{lrbox}
\usebox\lstbox

%\marginpar{\textcolor{red}{is \\ version 2 \\ correct?}}
 
\paragraph{Note} IOV of the residual error magnitude can also be encoded via variability 
on its parameters, a and b, similarly to the model described in section \ref{subsec:altIIV}.

%------------------------------------------------------------------------------------------------------------------------
\subsection{IRV  of the residual error magnitude}
The background information was given in section \ref{sec:variabStructure}. 
Here we cite from the first paper dealing with this model in the PK context.

\cite{Karlsson:1995rm} on inter-replicate variability: \emph{By taking
replicate measurements at a single time point it is possible to some extent
to differentiate between sources of error. Variability between replicates arises
from assay error and the error introduced by sample handling from the
point in the handling chain where the replication is made. We will assume
replication is made at the point of sampling. The difference between predicted 
and observed concentrations at the $j^{th}$ time point now has two components, 
a consistent difference, $\epsilon_{ij}$, between all replicates
and the prediction, and replicate-specific differences, $\epsilon_{ijk}$}.

%\begin{eqnarray}
%&&FLG =  \left\{ \begin{array}{rcll}  1 & \mbox{for}  & \mbox{TYPE}  = 2  \quad &; Replicate\;1\\
%0  & \mbox{for} & else  &; Replicate\;2 \nonumber
%\end{array}\right.
%\end{eqnarray}
%
%\begin{eqnarray}
%&&y_{ij} = f_{ij} + g\,\epsilon_{ij} \quad \mbox{with} \quad
%g = \sqrt{\sigma_1^2 + (1-FLG)^2\, \sigma_2^2 + FLG^2\, \sigma_3^2} \nonumber
%\end{eqnarray}
%
%\begin{eqnarray}
%&& y_{ij} = \left\{ \begin{array}{rcl}  f_{ij} + \epsilon_{1,ij} + \epsilon_{2,ij} & \mbox{for}  & \mbox{Replicate} = 1 \\
%f_{ij} + \epsilon_{1,ij}  + \epsilon_{3,ij}    & \mbox{for} & else  
%\end{array}\right. \quad \mbox{with} \quad \epsilon_{1,ij} \sim \mathcal{N}(0,\sigma_1^2), \quad \epsilon_{2,ij},\;\epsilon_{3,ij} \sim \mathcal{N}(0,\sigma^2) \nonumber 
%%&& \mbox{and} \quad g = \sqrt{\sigma_1^2 + \sigma^2} \nonumber
%\end{eqnarray}

\bigskip
Model definition:

\begin{eqnarray}
y_{ijk} = f_{ij} + \epsilon_{ij} + \epsilon_{ijk}; \quad 
\epsilon_{ij} \sim \mathcal{N}(0,\sigma_1^2), \; \epsilon_{ijk} \sim \mathcal{N}(0,\sigma_2^2); \quad var(y_{ijk}) = \sigma^2_{1} + \sigma^2_{2} \nonumber
\end{eqnarray}	

\bigskip
%% NONMEM
%\begin{lrbox}{\lstbox}\begin{minipage}{16cm}
%Original NMTRAN code:
%\begin{lstlisting}[frame=single,language=NM]
%$ERROR
%FLG = 0	; Replicate 1
%IF (TYPE.EQ.2) FLG = 1 ; Replicate 2
%Y = IPRED + EPS(1) + (1-FLG)*EPS(2) + FLG * EPS(3)
%$SIGMA 0.1 ; "Shared" error
%$SIGMA BLOCK (1) 0.05
%$SIGMA BLOCK (1) SAME
%\end{lstlisting}   
%\end{minipage}\end{lrbox}
%\usebox\lstbox

% NONMEM
\begin{lrbox}{\lstbox}\begin{minipage}{16cm}
NMTRAN (with two replicates)
\begin{lstlisting}[frame=single,language=NM]
$ERROR
IF (REP.EQ.1) IRV = EPS(2)
IF (REP.EQ.2) IRV = EPS(3)
Y = IPRED + EPS(1) + IRV
$SIGMA 0.1 ; "Shared" error
$SIGMA BLOCK (1) 0.05
$SIGMA BLOCK (1) SAME
\end{lstlisting}   
\end{minipage}\end{lrbox}
\usebox\lstbox

% MLXTRAN
\begin{lrbox}{\lstbox}\begin{minipage}{16cm}
MLXTRAN (with two replicates, indicated by different values in a column of type YTYPE)
\begin{lstlisting}[frame=single,language=MLX]
[LONGITUDINAL]
input = {..., a1, a2}

EQUATION:
F = ...
G = sqrt(a1^2 + a2^2)

DEFINITION:
Y1 = {distribution=normal, prediction=F, sd=G}
Y2 = {distribution=normal, prediction=F, sd=G}

\end{lstlisting}   
\end{minipage}\end{lrbox}
\usebox\lstbox

\paragraph{Note} IRV of the residual error magnitude can also be encoded via variability 
on its parameters, a1 and a2, similar to model described in section \ref{subsec:altIIV}.


%---------------------------------------------------------------------------3.4---------------------------------------------------------------------------------------
\subsection{Residual error magnitude varying with time-varying covariates/regressors}
\label{otherModels_model5}

Model definition example:

\begin{eqnarray}
&&MAT = 1/KA; \quad \text{mean absorption time} \nonumber \\
&&FACT = \left\{ \begin{array}{rcl}  1 & \mbox{if} \quad time > MAT \\
c & \mbox{otherwise}   
\end{array}\right. \nonumber \\
&&g = \sqrt{a^2 + b^2 f^2} \times FACT \nonumber
\end{eqnarray}

\bigskip
% NONMEM
\begin{lrbox}{\lstbox}\begin{minipage}{16cm}
NMTRAN
\begin{lstlisting}[frame=single,language=NM]
$ERROR
PROP = THETA(1)
ADD = THETA(2)
FACT = c
MAT = 1 / KA 
IF(TIME.GT.MAT) FACT = 1
W = SQRT(ADD**2+PROP**2*IPRED*IPRED)*FACT
Y = IPRED + W*EPS(1)
$THETA 1; PROP
	   1; ADD
$SIGMA 1 FIX
\end{lstlisting}   
\end{minipage}\end{lrbox}
\usebox\lstbox


% MLXTRAN
\begin{lrbox}{\lstbox}\begin{minipage}{16cm}
MLXTRAN
\begin{lstlisting}[frame=single,language=MLX]
[LONGITUDINAL]
input = {..., KA, a, b, c}

EQUATION:
MAT = 1/KA
if t > MAT
	FACT = 1
else
	FACT = c
end
F = ...
G = sqrt(a^2 + b^2*F^2)*FACT

DEFINITION:
Y = {distribution=normal, prediction=F, sd=G}
\end{lstlisting}   
\end{minipage}\end{lrbox}
\usebox\lstbox


%%%%%%%%%%%%%%%%%%%%%%%%%%%%%%%%%-2-%%%%%%%%%%%%%%%%%%%%%%%%%%%%%%%%%%%%%%%
\section{Joint residual error between multiple observations}
\label{otherModels_model2}

This models applies to joint  parent/metabolite measurements, discussed here 
in the specific case of constant error model for both observations\\
\bigskip
Model definition:
\begin{eqnarray}
&& y_{ij} = \left\{ \begin{array}{rcl}  f_{1,ij} + \epsilon_{1,ij} & \mbox{if}  & \mbox{TYPE}  = 1 \\
f_{2,ij} + \epsilon_{2,ij}    & \mbox{if} & \mbox{TYPE}  = 0  
\end{array}\right. \quad \mbox{with} \quad 
\bigg[ \begin{array}{l}  \epsilon_{1,ij} \\
\epsilon_{2,ij}   \nonumber
\end{array} \bigg]
\in \mathcal{N} 
\Bigg( 0, \bigg[ \begin{array}{ll} \sigma_1^2 & \sigma_{12} \\ \sigma_{12} & \sigma_2^2 \end{array}  \bigg] \Bigg)
\end{eqnarray}
where TYPE is the observation type.

\bigskip

% NONMEM
\begin{lrbox}{\lstbox}\begin{minipage}{16cm}
NMTRAN
\begin{lstlisting}[frame=single,language=NM]
$ERROR
IF (TYPE.EQ.1) THEN
Y = IPRED1 + EPS(1) ; Observation type 1
ELSE
Y = IPRED2 + EPS(2) ; Observation type 2
ENDIF
$SIGMA BLOCK(2)
0.1
0.05 0.1
\end{lstlisting}   
\end{minipage}\end{lrbox}
\usebox\lstbox

% MLXTRAN
\begin{lrbox}{\lstbox}\begin{minipage}{16cm}
MLXTRAN (with observation types indicated by different values in a column of type YTYPE)
\begin{lstlisting}[frame=single,language=MLX]
[LONGITUDINAL]
input = {..., a1, a2, r}

EQUATION:
F1 = ...
F2 = ...
G1 = a1
G2 = a2

DEFINITION:
Y1 = {distribution=normal, prediction=F1, sd=G1}
Y2 = {distribution=normal, prediction=F2, sd=G2}
correlation = {block={Y1,Y2} , corrcoef=r}

\end{lstlisting}   
\end{minipage}\end{lrbox}
\usebox\lstbox



%%-------------------------------------------------3.5.2-----------------------------------------------------------------------
%\subsection{case 2}
%
%\begin{eqnarray}
%&&TT = \theta_5 \nonumber \\
%&&FACT = \left\{ \begin{array}{rcl}  1 & \mbox{for}  & time > TT \\
%\theta_3  & \mbox{for} & else  \nonumber
%\end{array}\right.
%\end{eqnarray}
%
%\begin{eqnarray}
%&&g = \sqrt{\theta_2^2 + \theta_1^2 f^2} \times FACT \nonumber
%\end{eqnarray}
%
%\bigskip
%% NONMEM
%\begin{lrbox}{\lstbox}\begin{minipage}{16cm}
%Original NMTRAN code (extended by \$SIGMA and \$THETA):
%\begin{lstlisting}[frame=single,language=NM]
%$ERROR
%PROP = THETA(1)
%ADD = THETA(2)
%FACT = THETA(3)
%TT = THETA(4) ; Transition time
%IF(TIME.GT.TT) FACT = 1
%W = SQRT(ADD**2+PROP**2*IPRED*IPRED)*FACT
%Y = IPRED + W*EPS(1)
%$THETA 1; PROP
%	   1; ADD
%	   2; FACT
%	   10; TT
%$SIGMA 1 FIX
%\end{lstlisting}   
%\end{minipage}\end{lrbox}
%\usebox\lstbox
%
%% MLXTRAN
%\begin{lrbox}{\lstbox}\begin{minipage}{16cm}
%MLXTRAN
%\begin{lstlisting}[frame=single,language=MLX]
%[LONGITUDINAL]
%input = {S0, k, THETA1, THETA2, THETA3, TT}
%
%EQUATION:
%if t > TT
%	FACT = 1
%else
%	FACT = THETA3
%end
%F = S0*exp(-k*t)
%G = sqrt(THETA2^2 + THETA1^2*F^2)*FACT
%
%DEFINITION
%Y = {distribution=normal, prediction=F, sd=G}
%
%\end{lstlisting}   
%\end{minipage}\end{lrbox}
%\usebox\lstbox


%%---------------------------------------------3.5.3---------------------------------------------------------------------------
%\subsection{Case 2 -- covariate/time dependence}
%This model example, provided by Roberto, assumes a dependence between $g$ and both the regression variable, TIME, and a covariate, here \textit{subject} denoted as ID.
%
%\bigskip
%
%Model formulation: 
%\begin{eqnarray}
%y_{ij} = \left\{ \begin{array}{rcl}  f_{ij} + \epsilon_{1,ij}  & \mbox{for} & \mbox{TIME } \le 10 \;\;\&\;\; ID == 1 \\
%f_{ij} + \epsilon_{2,ij}  & \mbox{for} & \text{TIME} > 10 \;\;\&\;\; ID == 1 \nonumber \\
%\cdots \nonumber
%\end{array}\right\} \;\; \text{with} \;\;
%\mathit{var}(y_{ij}) = \left\{ \begin{array}{rcl}  \sigma_{1}^2  & \mbox{for} & \mbox{TIME} \le 10 \;\;\&\;\; ID == 1 \\
%\sigma_{2}^2  & \mbox{for} & \mbox{TIME} > 10 \;\;\&\;\; ID == 1  \nonumber \\
%\dots && \nonumber
%\end{array}\right.
%\end{eqnarray}
%for $\epsilon_{1,ij} \sim N(0,\sigma_1^2),\, \epsilon_{2,ij} \sim N(0,\sigma_2^2), \cdots$
%
%\bigskip
%% NONMEM
%\begin{lrbox}{\lstbox}\begin{minipage}{16cm}
%NONMEM
%\begin{lstlisting}[frame=single,language=NM]
%$ERROR 
%IPRED = F
%
%IF (TIME.LE.10.AND.ID.EQ.1) 
%Y = IPRED + EPS(1) 
%ENDIF
%
%IF (TIME.GT.10.AND.ID.EQ.1) 
%Y = IPRED + EPS(2)
%ENDIF
%...
%$SIGMA 1
%$SIGMA 1
%...
%\end{lstlisting}   
%\end{minipage}\end{lrbox}
%\usebox\lstbox
%
%
%% MLXTRAN
%\begin{lrbox}{\lstbox}\begin{minipage}{16cm}
%MLXTRAN
%\begin{lstlisting}[frame=single,language=MLX]
%[LONGITUDINAL]
%input = {S, k, a1, a2, ...}
%
%EQUATION:
%F = S*exp(-k*t)
%if (TIME <= 10 && ID == 1)
%	G = a1
%elseif (TIME > 10 && ID == 1)
%	G = a2
%elseif 
%	...
%end
%DEFINITION:
%Y = {distribution=normal, mean=F, sd=G}
%\end{lstlisting}   
%\end{minipage}\end{lrbox}
%\usebox\lstbox


%------------------------------------------------2.1.8-----------------------------------------------------------------------
\section{Two (or more) types of measurements error model}
\label{model8}
This model assumes that e.g. the drug concentration, is measured using  different assays. 
This is expressed in the following example by the categorical variable ASY taking values 
1 or 0, for the first or second assay respectively (\cite{NONMEM:2006aa}).  \\\	
\bigskip
Model definition:
\begin{eqnarray}
&& y_{ij} = f_{ij} + \text{ASY}_j\epsilon_{1,ij} + (1-\text{ASY}_j) \epsilon_{2,ij}; \quad \epsilon_{1,ij} \sim N(0,\sigma_1^2); \quad \epsilon_{2,ij} \sim N(0,\sigma_2^2); \nonumber
\end{eqnarray}
Alternative formulation:
\begin{eqnarray}
&& y_{ij} = \left\{ \begin{array}{rcl}  f_{ij} + \epsilon_{1,ij}  & \mbox{if} & \text{ASY}_j  = 1 \\
f_{ij} + \epsilon_{2,ij}  & \mbox{if} & \text{ASY}_j  = 0 \nonumber
\end{array}\right.;  \quad \epsilon_{1,ij} \sim N(0,\sigma_1^2); \quad \epsilon_{2,ij} \sim N(0,\sigma_2^2); 
\end{eqnarray}
In both formulations:
\begin{eqnarray}
&& \mathit{var}(y_{ij}) = \left\{ \begin{array}{rcl}  \sigma_{1}^2  & \mbox{if} & \text{ASY}_j  = 1 \\
\sigma_{2}^2  & \mbox{if} & \text{ASY}_j  = 0 \nonumber
\end{array}\right.
\end{eqnarray}

\bigskip
% NONMEM
\begin{lrbox}{\lstbox}\begin{minipage}{16cm}
NMTRAN
\begin{lstlisting}[frame=single,language=NM]
$ERROR
IPRED=F
IF (ASY.EQ.1) Y = IPRED + EPS(1)
IF (ASY.EQ.0) Y = IPRED + EPS(2)
$SIGMA 1
$SIGMA 1
\end{lstlisting}   
\end{minipage}\end{lrbox}
\usebox\lstbox


% MLXTRAN
\begin{lrbox}{\lstbox}\begin{minipage}{16cm}
MLXTRAN (with YTYPE for the column ASY)
\begin{lstlisting}[frame=single,language=MLX]
[LONGITUDINAL]
input = {a1, a2, ...}

EQUATION:
F = ... 
G1 = a1
G2 = a2

DEFINITION:
Y1 = {distribution=normal, mean=F, sd=G1}
Y2 = {distribution=normal, mean=F, sd=G2}
\end{lstlisting}   
\end{minipage}\end{lrbox}
\usebox\lstbox


